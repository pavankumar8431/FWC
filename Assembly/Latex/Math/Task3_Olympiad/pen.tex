\documentclass[12pt,a4paper]{article}

\usepackage{amsmath,amssymb}
\usepackage{geometry}
\usepackage{graphicx}
\usepackage{fancyhdr}

\geometry{margin=1in}
\setlength{\headheight}{60pt}

\fancypagestyle{firstpage}{
\fancyhf{}
\lhead{
\IfFileExists{Logo.jpg}
{\includegraphics[height=1.8cm]{Logo.jpg}}
{}
}
\rhead{
\textbf{Pavan Kumar KS}\\
ID: COMETFWC044\\
Date: \today
}
\renewcommand{\headrulewidth}{0.4pt}
}

\pagestyle{plain}

\begin{document}

\thispagestyle{firstpage}

\begin{center}
{\Large\bfseries Twenty-second International Olympiad, 1981}
\end{center}

\bigskip

\textbf{1981/1.}
$P$ is a point inside a given triangle $ABC$. $D, E, F$ are the feet of the
perpendiculars from $P$ to the lines $BC, CA, AB$ respectively.
Find all $P$ for which
\[
\frac{BC}{PD} + \frac{CA}{PE} + \frac{AB}{PF}
\]
is least.

\medskip

\textbf{1981/2.}
Let $1 \le r \le n$ and consider all subsets of $r$ elements of the set
$\{1,2,\ldots,n\}$. Each of these subsets has a smallest member.
Let $F(n,r)$ denote the arithmetic mean of these smallest numbers; prove that
\[
F(n,r) = \frac{n+1}{r+1}.
\]

\medskip

\textbf{1981/3.}
Determine the maximum value of $m^3 + n^3$, where $m$ and $n$ are integers
satisfying $m,n \in \{1,2,\ldots,1981\}$ and
\[
(n^2 - mn - m^2)^2 = 1.
\]

\medskip

\textbf{1981/4.}
\begin{enumerate}
\item[(a)] For which values of $n>2$ is there a set of $n$ consecutive positive
integers such that the largest number in the set is a divisor of the least
common multiple of the remaining $n-1$ numbers?
\item[(b)] For which values of $n>2$ is there exactly one set having the stated
property?
\end{enumerate}

\medskip

\textbf{1981/5.}
Three congruent circles have a common point $O$ and lie inside a given triangle.
Each circle touches a pair of sides of the triangle. Prove that the incenter
and the circumcenter of the triangle and the point $O$ are collinear.

\medskip

\textbf{1981/6.}
The function $f(x,y)$ satisfies
\[
\begin{aligned}
(1)\;& f(0,y) = y+1,\\
(2)\;& f(x+1,0) = f(x,1),\\
(3)\;& f(x+1,y+1) = f(x,f(x+1,y)),
\end{aligned}
\]
for all non-negative integers $x,y$. Determine $f(4,1981)$.

\newpage

\begin{center}
{\Large\bfseries Twenty-third International Olympiad, 1982}
\end{center}

\bigskip

\textbf{1982/1.}
The function $f(n)$ is defined for all positive integers $n$ and takes on
non-negative integer values. Also, for all $m,n$,
\[
f(m+n) - f(m) - f(n) = 0 \text{ or } 1,
\]
$f(2)=0$, $f(3)>0$, and $f(9999)=3333$. Determine $f(1982)$.

\medskip

\textbf{1982/2.}
A non-isosceles triangle $A_1A_2A_3$ is given with sides $a_1,a_2,a_3$
($a_i$ opposite $A_i$). For all $i=1,2,3$, $M_i$ is the midpoint of side $a_i$,
and $T_i$ is the point where the incircle touches side $a_i$.
Denote by $S_i$ the reflection of $T_i$ in the interior bisector of angle $A_i$.
Prove that the lines $M_1S_1$, $M_2S_2$, $M_3S_3$ are concurrent.

\medskip

\textbf{1982/3.}
Consider the infinite sequences $\{x_n\}$ of positive real numbers with
\[
x_0 = 1, \quad x_{i+1} \le x_i \text{ for all } i \ge 0.
\]
\begin{enumerate}
\item[(a)] Prove that for every such sequence there is an $n \ge 1$ such that
\[
\frac{x_0^2}{x_1} + \frac{x_1^2}{x_2} + \cdots + \frac{x_{n-1}^2}{x_n} \ge 3.999.
\]
\item[(b)] Find such a sequence for which
\[
\frac{x_0^2}{x_1} + \frac{x_1^2}{x_2} + \cdots + \frac{x_{n-1}^2}{x_n} < 4.
\]
\end{enumerate}

\medskip

\textbf{1982/4.}
Prove that if $n$ is a positive integer such that the equation
\[
x^3 - 3xy^2 + y^3 = n
\]
has a solution in integers $(x,y)$, then it has at least three such solutions.
Show that the equation has no solutions in integers when $n=2891$.

\medskip

\textbf{1982/5.}
The diagonals $AC$ and $CE$ of the regular hexagon $ABCDEF$ are divided by the
inner points $M$ and $N$ respectively, so that
\[
\frac{AM}{AC} = \frac{CN}{CE} = r.
\]
Determine $r$ if $B,M,N$ are collinear.

\medskip

\textbf{1982/6.}
Let $S$ be a square with sides of length $100$, and let $L$ be a path within $S$
which does not meet itself and which is composed of line segments
$A_0A_1,A_1A_2,\ldots,A_{n-1}A_n$ with $A_0 \ne A_n$.
Suppose that for every point $P$ of the boundary of $S$ there is a point of $L$
at distance from $P$ not greater than $1/2$.
Prove that there are two points $X$ and $Y$ in $L$ such that the distance between
$X$ and $Y$ is not greater than $1$, and the length of that part of $L$ which lies
between $X$ and $Y$ is not smaller than $198$.

\end{document}
